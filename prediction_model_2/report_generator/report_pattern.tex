\documentclass{article}
\usepackage{graphicx}
\usepackage{xcolor}
\usepackage{colortbl}
\usepackage{float}
\usepackage[T1]{fontenc}

\title{Raport Plagiat}
\date{}

\begin{document}

\maketitle

\begin{center}
    \textbf{Data:} Input-nr-0 \\
    \textbf{Czas:} Input-nr-1 \\
    \textbf{Nazwa testowanego pliku:} Input-nr-2
\end{center}

\section{Inforacje ogólne dotyczące testu}

\textbf{Wynik procentowy:} \textcolor{Input-nr-3}{ Input-nr-4 }

\textbf{Czas trwania testu:} Input-nr-5

\textbf{Liczna elementów w teście:} Input-nr-6


\section{Images}

Below is an image that is placed here using the \texttt{[H]} option to force the image placement.

\begin{figure}[H]
\centering
\includegraphics[width=0.5\textwidth]{ Plot-nr-0 }
\caption{An example image}
\end{figure}

\section{Highlighted Text}

\begin{figure}[H]
\centering
\includegraphics[width=0.5\textwidth]{ Plot-nr-1 }
\caption{An example image}
\end{figure}


Here is some \textcolor{yellow}{highlighted text} to demonstrate how to use colors in LaTeX.

\section{Text}

This is some regular text to demonstrate basic formatting in LaTeX. You can add paragraphs, sections, and subsections easily.

\section{Table}

Below is an example of a table with colored cells:

\begin{table}[H]
\centering
\begin{tabular}{|c|c|c|}
\hline
\rowcolor{gray!30} \textbf{Header 1} & \textbf{Header 2} & \textbf{Header 3} \\ \hline
\textcolor{red}{Cell 1}              & Cell 2            & Cell 3            \\ \hline
Cell 4                               & \textcolor{blue}{Cell 5} & Cell 6            \\ \hline
Cell 7                               & Cell 8            & \textcolor{green}{Cell 9}  \\ \hline
\end{tabular}
\caption{Example Table with Colored Cells}
\end{table}


\end{document}
