\documentclass{article}
\usepackage{graphicx}
\usepackage{xcolor}
\usepackage{colortbl}
\usepackage{float}
\usepackage[T1]{fontenc}

\title{Raport Plagiat}
\date{}

\begin{document}

\maketitle

\begin{center}
    \textbf{Data:} Input-date \\
    \textbf{Czas:} Input-hour \\
    \textbf{Nazwa testowanego pliku:} Input-file_name
\end{center}

\section{Inforacje ogólne dotyczące testu}

\textbf{Wynik procentowy:} Input-proc_plagiat

\textbf{Czas trwania testu:} Input-exec_time

\textbf{Liczna elementów w teście:} Input-nr-6


\section{Images}

Below is an image that is placed here using the \texttt{[H]} option to force the image placement.

\begin{figure}[H]
\centering
\includegraphics[width=1\textwidth]{ Plot-nr-0 }
\caption{An example image}
\end{figure}

\section{Highlighted Text}

\begin{figure}[H]
\centering
\includegraphics[width=1\textwidth]{ Plot-nr-1 }
\caption{An example image}
\end{figure}




\end{document}
