
\documentclass[a4paper,10pt]{beamer}
\usepackage[T1,plmath]{polski}
\usepackage[cp1250]{inputenc}
\usepackage{amssymb}
\usepackage{indentfirst}
\usepackage{graphicx}

\usefonttheme[onlymath]{serif}


\usepackage{ulem} % kolorowe podkreślenia
\usepackage{xcolor} % kolorowe podkreślenia

\usepackage{diagbox}
\usepackage{tasks}

\newcommand{\arcctg}{{\rm{arcctg}\,}}
\newcommand{\arctg}{{\rm{arctg}\,}}
\newcommand{\ddd}{{\,\rm{d}}}
\newcommand{\Int}{{\,\rm{Int}}}

%\definecolor{green1}{html}{22B14C}

\newcommand{\ouline}[1]{{\color{orange}\uline{{\color{black}#1}}}} % pomarańczowe podkreślenie
\newcommand{\yuline}[1]{{\color{yellow}\uline{{\color{black}#1}}}} % żółte podkreślenie
\newcommand{\buline}[1]{{\color{blue}\uline{{\color{black}#1}}}} % niebieskie podkreślenie
\newcommand{\guline}[1]{{\color[RGB]{34,177,76}\uline{{\color{black}#1}}}} % zielone podkreślenie


\usetheme{Boadilla}
\usecolortheme{crane}
%\usecolortheme[rgb={1,0.5,0}]{structure}

\title{\bf Elementy teorii liczb}
%\subtitle{Matematyka, Kierunek: Architektura}
\author[B. Pawlik]{\bf dr inż. Bartłomiej Pawlik}
%\institute{}



%\setbeamercovered{transparent} % przezroczyste warstwy





\begin{document}




\begin{frame}
\titlepage
\end{frame}






\begin{frame}{Słowniczek}
	
	\begin{tabular}{cl}
		$|x|$&wartość bezwzględna liczby $x$\\
		$\mbox{NWD}(a,b)$&największy wspólny dzielnik liczb $a$ i $b$\\
		$\mbox{NWW}(a,b)$&najmniejsza wspólna wielokrotność liczb $a$ i $b$\\
		$\min\{a,b\}$&niewiększa z liczb $a$ i $b$\\
		$\max\{a,b\}$&niemniejsza z liczb $a$ i $b$\\
		$a|b$&liczba $a$ jest dzielnikiem liczby $b$\\
		$a\perp b$&liczby $a$ i $b$ są względnie pierwsze\\
		$\mathbb{N}$&zbiór liczb naturalnych, $\mathbb{N}=\{1,2,3,\ldots\}$\\	
		$\mathbb{Z}$&zbiór liczb całkowitych, $\mathbb{Z}=\{\ldots,-2,-1,0,1,2,\ldots\}$\\
		$\mathbb{Z}_n$&zbiór reszt z dzielenia przez $n$, $\mathbb{Z}_n=\{0,1,\ldots,n-1\}$\\
		$\mathbb{P}$&zbiór liczb pierwszych	\\
		$p_i$&$i$-ta liczba pierwsza
	\end{tabular}
	
\end{frame}





\begin{frame}
	
	\begin{block}{Definicja NWD}
		Niech $a,b\in\mathbb{Z}$ i niech co najmniej jedna z nich jest różna od $0$. Liczbę naturalną $d$ nazywamy {\bf największym wspólnym dzielnikiem} liczb $a$ i $b$, gdy
		\begin{itemize}
			\item $d|a$ i $d|b$,
			\item jeżeli dla $c\in\mathbb{N}$ mamy $c|a$ i $c|b$, to $c|d$.
		\end{itemize}
	\end{block}

Największy wspólny dzielnik liczb $a$ i $b$ oznaczamy jako $\mbox{NWD}(a,b)$.

	\begin{exampleblock}{Przykład}
		$\mbox{NWD}(6,8)=2,\ \ \mbox{NWD}(14,-17)=1,\ \ \mbox{NWD}(-3,-9)=3,\ \ \mbox{NWD}(0,24)=24.$
	\end{exampleblock}
	
\end{frame}




\begin{frame}
	
	\begin{block}{Definicja NWW}
		Niech $a,b\in\mathbb{Z}/\{0\}$. Liczbę $D\in\mathbb{N}$ nazywamy {\bf najmniejszą wspólną wielokrotnością} liczb $a$ i $b$, gdy
		\begin{itemize}
			\item $a|D$ i $b|D$,
			\item jeżeli dla $c\in\mathbb{N}$ mamy $a|c$ i $b|c$, to $D|c$.
		\end{itemize}
	\end{block}
	
	Najmniejszą wspólną wielokrotność liczb $a$ i $b$ oznaczamy jako $\mbox{NWW}(a,b)$.
	
	\begin{exampleblock}{Przykład}
		$\mbox{NWW}(6,8)=24,\ \ \mbox{NWW}(14,-17)=238,\ \ \mbox{NWW}(-3,-9)=9.$
	\end{exampleblock}

	\begin{block}
				
		W literaturze często można spotkać się z oznaczeniami $\mbox{NWD}(a,b)=(a,b)$ i~$\mbox{NWW}(a,b)=[a,b]$.		
	\end{block}
	
\end{frame}



\begin{frame}
	
	\begin{block}{Własności NWD i NWW}
		Niech $a,b\in\mathbb{Z}/\{0\}$ i $q\in\mathbb{Z}$.
		\begin{enumerate}
			\item Jeżeli $a|b$, to $\mbox{NWD}(a,b)=|a|$ i $\mbox{NWW}(a,b)=|b|$.
			\item $\mbox{NWD}(a,b)=\mbox{NWD}(|a|,|b|)$ i $\mbox{NWW}(a,b)=\mbox{NWW}(|a|,|b|)$.
			\item $\mbox{NWD}(a,b)=\mbox{NWD}(a-qb,b)$.
			\item\label{mno} $\mbox{NWD}(a,b)\cdot\mbox{NWW}(a,b)=|a\cdot b|.$
		\end{enumerate}
	\end{block}
	
\end{frame}


\begin{frame}{Wyznaczenie NWW i NWD}
	
	\begin{block}{Twierdzenie}
		Niech $$a=p_{1}^{\alpha_{1}}\cdot p_{2}^{\alpha_{2}}\cdot\ldots\cdot p_{k}^{\alpha_{k}}\ \ \mbox{ oraz}\ \ b=p_{1}^{\beta_{1}}\cdot p_{2}^{\beta_{2}}\cdot\ldots\cdot p_{k}^{\beta_{k}}.$$
		Wtedy
		$$NWD(a,b)=p_{1}^{\min\{\alpha_1,\beta_1\}}\cdot p_{2}^{\min\{\alpha_2,\beta_2\}}\cdot\ldots\cdot p_{k}^{\min\{\alpha_k,\beta_k\}}$$
		oraz
		$$NWW(a,b)=p_{1}^{\max\{\alpha_1,\beta_1\}}\cdot p_{2}^{\max\{\alpha_2,\beta_2\}}\cdot\ldots\cdot p_{k}^{\max\{\alpha_k,\beta_k\}}.$$	
	\end{block}

\end{frame}

\begin{frame}
	
	\begin{exampleblock}{Przykład}
		Wyznaczyć największy wspólny dzielnik oraz najmniejszą wspólną wielokrotność liczb $48$ i $180$.
		
		Zauważmy, że $48=2^4\cdot3$ oraz $180=2^2\cdot3^2\cdot5$.
		
		Zatem $$\displaystyle\mbox{NWD}(48,180)=2^{\min\{4,2\}}\cdot3^{\min\{1,2\}}\cdot5^{\min\{0,1\}}=2^2\cdot3^1\cdot5^0=12$$
		oraz $$\displaystyle\mbox{NWW}(48,180)=2^{\max\{4,2\}}\cdot3^{\max\{1,2\}}\cdot5^{\max\{0,1\}}=2^4\cdot3^2\cdot5^1=720.$$
		
	\end{exampleblock}

Przedstawiona metoda nie jest efektywna (ponieważ wymaga rozkładu na czynniki). NWD można obliczyć szybciej korzystając z algorytmu Euklidesa.
	
\end{frame}

\begin{frame}{Algorytm Euklidesa dla NWD}
	
	\begin{block}{}
	
	Niech $a,b\in\mathbb{N}$ i $a>b$.
	
	Po podzieleniu z resztą $a$ przez $b$ otrzymujemy $a=q_1b+r_1$.
	
	Jeżeli $r_1=0$, to $\mbox{NWD}(a,b)=b$.
	
	Jeżeli $r_1\neq0$, to dzielimy z resztą $b$ przez $r_1$ i otrzymujemy $b=q_2r_1+r_2$.
	
	Procedura kończy się, gdy dla pewnego indeksu $n$ mamy $r_n\neq0$ oraz $r_{n+1}=0$. Wtedy $NWD(a,b)=r_n$.
	
	\end{block}

	\begin{align*}
		a&=q_1\cdot b+r_1&0<&\,r_1<b\\
		b&=q_2\cdot r_1+r_2&0<&\,r_2<r_1\\
		r_1&=q_3\cdot r_2+r_3&0<&\,r_3<r_2\\
		&\ \, \vdots&&\ \,\vdots\\
		r_{n-2}&=q_n\cdot r_{n-1}+r_n&0<&\,r_n<r_{n-1}\\
		r_{n-1}&=q_{n+1}\cdot \textcolor{red}{r_n}\,\textcolor{gray}{+\,0}&&
	\end{align*}
$$\mbox{NWD}(a,b)=\textcolor{red}{r_n}$$

\end{frame}


\begin{frame}
	
	\begin{exampleblock}{Przykład}
		Wyznaczyć największy wspólny dzielnik oraz najmniejszą wspólną wielokrotność liczb $48$ i $180$.
		
		Stosując algorytm Euklidesa otrzymujemy
				\begin{align*}
			180&=3\cdot48+36\\
			48&=1\cdot36+12\\
			36&=3\cdot\textcolor{red}{12}
		\end{align*}
		Zatem $\mbox{NWD}(48,180)=12$.
		
		Z faktu $\mbox{NWD}(a,b)\cdot\mbox{NWW}(a,b)=|a\cdot b|$ otrzymujemy $\displaystyle\mbox{NWW}(a,b)=\frac{|a\cdot b|}{\mbox{NWD}(a,b)}$.
		
		Zatem $$NWW(48,180)=\frac{48\cdot180}{12}=\frac{4\cdot180}{1}=720.$$
		
	\end{exampleblock}

\end{frame}



\begin{frame}
	
	\begin{block}{Poprawność algorytmu Euklidesa}
		\begin{itemize}
			\item Algorytm produkuje \underline{malejący} ciąg liczb całkowitych nieujemnych $r_1>r_2>\ldots>r_n$ (jedna liczba w jednym kroku). Zatem algorytm zatrzymuje się po skończonej liczbie kroków (nie większej niż wartość $r_1$). 
			\item Z własności $\mbox{NWD}(a,b)=\mbox{NWD}(a-qb,b)$ otrzymujemy
			\begin{align*}
			\mbox{NWD}(a,b)&=\mbox{NWD}(b,r_1)=\mbox{NWD}(r_1,r_2)=\ldots=\mbox{NWD}(r_{n-1},r_n)=\\
			&=\mbox{NWD}(r_n,0)=r_n
			\end{align*}
		\end{itemize}
	\end{block}

\end{frame}


\begin{frame}

	\begin{block}{Twierdzenie (NWD jako kombinacja liniowa)}
		Dla $a,b\in\mathbb{Z}$ takich, że co najmniej jedna z nich jest różna od 0, istnieją $u,v\in\mathbb{Z}$ takie, że
		$$\mbox{NWD}(a,b)=u\cdot a+v\cdot b.$$
		Ponadto $\mbox{NWD}(a,b)$ jest najmniejszą możliwą \underline{dodatnią} kombinacją liniową $a$ i $b$.
	\end{block}

	\begin{exampleblock}{Przykład}
		Wyznaczyć najmniejszą dodatnią kombinację liniową liczb $3$ i $7$ oraz podać jej przykładowe współczynniki.
		\begin{align*}
		\textcolor{gray}{\mbox{NWD}(3,7)=\,}1&=5\cdot3-2\cdot7\\
		1&=(-2)\cdot3+1\cdot7\\
		1&=(-23)\cdot3+10\cdot7
		\end{align*}
	\end{exampleblock}

\end{frame}



\begin{frame}{Odwrotny algorytm Euklidesa}
	
	Algorytm służy wyznaczenia $u$ i $v$ takich, że $a\cdot u+b\cdot v=\mbox{NWD}(a,b)$.
	
	\begin{block}{}
		\begin{itemize}
			\item Obliczamy $\mbox{NWD}(a,b)$ korzystając z algorytmu Euklidesa otrzymując ciąg równań
			
			$a=q_1\cdot b+r_1,\ \ \ b=q_2\cdot r_1+r_2,\ \ \ r_1=q_3\cdot r_2+r_3,\ \ \ \ldots,$
			
			$r_{n-3}=q_{n-1}\cdot r_{n-2}+r_{n-1},\ \ \ r_{n-2}=q_n\cdot r_{n-1}+r_n,\ \ \ r_{n-1}=q_{n+1}\cdot r_n.$
			
			\item Z $i$-tego równania wyznaczamy wartość $r_i$ dla każdego $i=1,2,\ldots,n$ (więc pomijamy ostatnie równanie).
			
			\item Wyliczone $r_n$ daje nam równanie
			
			$\mbox{NWD}(a,b)=r_{n-2}-q_n\cdot r_{n-1}$. 
			
			Do tego równania wstawiamy wyliczoną wartość $r_{n-1}$ (w ten sposób otrzymujemy $\mbox{NWD}(a,b)$ w kombinacji liniowej $r_{n-2}$ i $r_{n-3}$).
			
			\item Kontynuujemy podstawianie $r_{n-2}$, $r_{n-3}$ itd. aż do $r_1$, po drodze upraszczając współczynniki. W efekcie dostajemy zapis implikujący wartości $u$ i $v$.
		\end{itemize}
	\end{block}
	
\end{frame}






\begin{frame}
	
	\begin{block}{Definicja}
		Liczba $n\in\mathbb{N}$ jest {\bf liczbą pierwszą}, jeżeli $n$ ma dokładnie dwa dodatnie dzielniki. 
	\end{block}


		\begin{itemize}
		\item $0$ nie jest liczbą pierwszą (po pierwsze nie jest liczbą dodatnią, a po drugie ma nieskończenie wiele dzielników).
		
		\item $1$ nie jest liczbą pierwszą (ma dokładnie jeden dodatni dzielnik).
		
		\item Początkowe liczby pierwsze: $2,\,3,\,5,\,7,\,11,\,13,\,17,\,19,\,23$.
		
		\item Liczby naturalne \underline{większe od $1$} dzielimy na liczby pierwsze i liczby złożone (złożone to te, które nie są pierwsze). 
		
		\item $1$ nie jest ani liczbą pierwszą, ani liczbą złożoną.
		
		\item Zbiór liczb pierwszych oznaczamy przez $\mathbb{P}$.
		\end{itemize}
	

\end{frame}

\begin{frame}
	
	\begin{block}{Twierdzenie}
		Liczb pierwszych jest nieskończenie wiele.
	\end{block}
	
	\begin{proof}
		Załóżmy nie wprost, że teza twierdzenia jest fałszywa, tj. zbiór liczb pierwszych jest skończony. Zatem dla pewnej liczby naturalnej $n$ mamy $$\mathbb{P}=\{p_1,\,p_2,\,\ldots,\,p_n\}.$$ Niech $P$ będzie następnikiem iloczynu wszystkich elementów powyższego zbioru $\mathbb{P}$:
		$$P=1+\prod\limits_{i=1}^n{p_i}.$$
		Zauważmy, że liczba $P$ przy dzieleniu przez $p_i$ (dla $i=1,2,\ldots,n$) daje resztę $1$, zatem liczba $P$ nie jest podzielna przez żadną liczbę pierwszą --- uzyskaliśmy sprzeczność.
	\end{proof}
	Powyższy dowód ma $\sim$2500 lat ({\it Elementy} Euklidesa).
	
\end{frame}

\begin{frame}
	Czy z powyższego dowodu wynika, że liczba $P$ jest liczbą pierwszą? Nie!
	\begin{align*}
2+1&=3\in\mathbb{P}\\
2\cdot3+1&=7\in\mathbb{P}\\
2\cdot3\cdot5+1&=31\in\mathbb{P}\\
2\cdot3\cdot5\cdot7+1&=211\in\mathbb{P}\\
2\cdot3\cdot5\cdot7\cdot11+1&=2311\in\mathbb{P}\\
2\cdot3\cdot5\cdot7\cdot11\cdot13+1&=59\cdot509\\
2\cdot3\cdot5\cdot7\cdot11\cdot13\cdot17+1&=19\cdot97\cdot277\\
2\cdot3\cdot5\cdot7\cdot11\cdot13\cdot17\cdot19+1&=347\cdot27\,953\\
2\cdot3\cdot5\cdot7\cdot11\cdot13\cdot17\cdot19\cdot23+1&=317\cdot703\,763\\
2\cdot3\cdot5\cdot7\cdot11\cdot13\cdot17\cdot19\cdot23\cdot29+1&=331\cdot571\cdot34\,231\\
2\cdot3\cdot5\cdot7\cdot11\cdot13\cdot17\cdot19\cdot23\cdot29\cdot31+1&=200\,560\,490\,131\in\mathbb{P}\\
2\cdot3\cdot5\cdot7\cdot11\cdot13\cdot17\cdot19\cdot23\cdot29\cdot31\cdot37+1&=181\cdot60\,611\cdot676\,421\\
	\end{align*}
	Zatem konstrukcja w powyższym dowodzie nie daje przepisu na tworzenie coraz większych liczb pierwszych, a jedynie wskazuje, że istnieją liczby pierwsze nienależące do dowolnego skończonego zbioru liczb pierwszych.
\end{frame}
	
\begin{frame}
Liczby o jeden większe od iloczynu początkowych liczb pierwszych to tzw. {\it liczby Euklidesa}:
$$3,\,7,\,31,\,211,\,2311,\,30\,031,\,510\,511,\ldots$$

\begin{block}{Definicja}
Liczbę $$E_n=1+\prod\limits_{i=1}^np_i$$ nazywamy {\it $n$-tą liczbą Euklidesa}.
\end{block}

\bigskip

\begin{block}{}
Do dzisiaj nie wiadomo czy
\begin{itemize}
\item jest nieskończenie wiele liczb pierwszych Euklidesa?
\item każda liczba Euklidesa jest bezkwadratowa?
\end{itemize}
\end{block}
\end{frame}


\begin{frame}
	
	\begin{block}{Podstawowe twierdzenie arytmetyki}
		Każdą liczbę całkowitą dodatnią można przedstawić jako iloczyn liczb pierwszych. Przedstawienie takie jest jednoznaczne z dokładnością do kolejności czynników.
	\end{block}

	\begin{exampleblock}{Przykład}
		$12=2\cdot2\cdot3=2\cdot3\cdot2=3\cdot2\cdot2$
	\end{exampleblock}
	
	\begin{block}{Wniosek}
		Każda większa od $1$ liczba naturalna $n$ może być jednoznacznie zapisana w tzw. {\bf postaci kanonicznej}
		$$n=q_{1}^{\alpha_{1}}\cdot q_{2}^{\alpha_{2}}\cdot\ldots\cdot q_{k}^{\alpha_{k}},$$
		gdzie $q_{i}$ są liczbami pierwszymi, $\alpha_i$ są liczbami naturalnymi oraz $q_1<q_2<\ldots<q_k$.
	\end{block}

	\begin{exampleblock}{Przykład}
	Postacią kanoniczną liczby $12$ jest $2^2\cdot3$.
	\end{exampleblock}
	
\end{frame}




\begin{frame}{Funkcja $\varphi$-Eulera}
	
	\begin{block}{Definicja}
		Liczby całkowite $a$ i $b$ nazywamy {\bf względnie pierwszymi}, gdy $\mbox{NWD}(a,b)=1$. 
	\end{block}
	
	\begin{block}{}
		Zapis $a\perp b$ oznacza, że $a$ i $b$ są liczbami względnie pierwszymi.
	\end{block}

	\begin{block}{Stwierdzenie}
		Liczby $\displaystyle\frac{a}{\mbox{NWD}(a,b)}$ i $\displaystyle\frac{b}{\mbox{NWD}(a,b)}$ są względnie pierwsze.
	\end{block}

	\begin{exampleblock}{Przykład}
		$\mbox{NWD}(48,180)=12$, więc $48$ i $180$ nie są liczbami względnie pierwszymi.
		$$\frac{48}{12}=4,\ \ \ \frac{180}{12}=15.$$
		Zauważmy, że $\mbox{NWD}(4,15)=1$. Zatem $4\perp15$.
	\end{exampleblock}
	
\end{frame}


\begin{frame}
	
	\begin{block}{Definicja}
		Dla każdej liczby $n\in\mathbb{N}/\{1\}$ określamy liczbę $\varphi(n)$ jako liczbę dodatnich liczb całkowitych mniejszych od $n$ i względnie pierwszych z $n$:
		
		$$\varphi(n)=\Big|\{1\leqslant k<n:\,k\perp n \}\Big|.$$
		
		 Funkcję $\varphi=\varphi(n)$ nazywamy {\bf funkcją $\varphi$-Eulera}.
	\end{block}
	
	
	\begin{exampleblock}{Przykład}
		Obliczmy $\mbox{NWD}(k,12)$ dla $k$ mniejszych od 12:
			$$\begin{array}{llll}
			\mbox{NWD}(1,12)=\textcolor{red}{1}, &\mbox{NWD}(4,12)=4, &\mbox{NWD}(7,12)=\textcolor{red}{1},&\mbox{NWD}(10,12)=2, \\
			\mbox{NWD}(2,12)=2, &\mbox{NWD}(5,12)=\textcolor{red}{1}, &\mbox{NWD}(8,12)=4,&\mbox{NWD}(11,12)=\textcolor{red}{1}. \\
			\mbox{NWD}(3,12)=3, &\mbox{NWD}(6,12)=6, &\mbox{NWD}(9,12)=3,
			\end{array}$$
		Zatem
		$$\varphi(12)=\Big|\{1,5,7,11\}\Big|=4.$$

	\end{exampleblock}	
	
	
	
\end{frame}	
	
	
\begin{frame}


\begin{block}{Stwierdzenie}
 Dla dowolnej liczby pierwszej $p$ i liczby całkowitej dodatniej $\alpha$ zachodzi:
 \begin{itemize}
 \item $\varphi(p)=p-1$,\\
 \item $\varphi(p^{\alpha})=p^{\alpha}\cdot\left(1-\frac1p\right).$
 \end{itemize}
\end{block}

\begin{proof}
	\begin{itemize}
		\item Liczba pierwsza $p$ jest  względnie pierwsza z każdą z liczb $1,2,\ldots,p-1$.
		\item Zauważmy, że jedynie wielokrotności liczby pierwszej $p$ mają wspólny nietrywialny dzielnik z $p^\alpha$. Zatem w zbiorze $\{1,2,\ldots,p^\alpha-1\}$ liczbami niebędącymi liczbami względnie pierwszymi z $p^\alpha$ są
		$$1\cdot p,\,2\cdot p,\,\ldots,\,(p^{\alpha-1}-1)\cdot p,$$
		więc ich liczba wynosi $p^{\alpha-1}-1$.
		Zatem
		$$\varphi\left(p^{\alpha}\right)=\left(p^{\alpha}-1\right)-\left(p^{\alpha-1}-1\right)=p^{\alpha}-p^{\alpha-1}=p^{\alpha}\left(1-\frac1p\right).$$
	\end{itemize}
\end{proof}


\end{frame}


\begin{frame}

	\begin{block}{Stwierdzenie}
	
	Jeżeli $a\perp b$, to $\varphi(ab)=\varphi(a)\varphi(b)$. %dowodzi się z Chińskiego twierdzenia o resztach
	
	\end{block}
	
	Z dwóch ostatnich stwierdzeń wynika następujące

	\begin{block}{Twierdzenie}
		Niech $\displaystyle p_{1}^{\alpha_{1}}\cdot p_{2}^{\alpha_{2}}\cdot\ldots\cdot p_{k}^{\alpha_{k}}$ będzie postacią kanoniczną liczby $n\in\mathbb{N}/\{1\}$. Wtedy
		$$\varphi(n)=n\cdot\prod\limits_{i=1}^k\left(1-\frac1{p_i}\right).$$
	\end{block}
	
\end{frame}




\begin{frame}
	
	\begin{exampleblock}{Przykład}
		Wyznaczyć liczby $\varphi(180)$ i $\varphi(12\,936)$.
		\begin{itemize}
			\item Postać kanoniczna liczby $180$ to $2^2\cdot3^2\cdot5$, więc jedynymi dzielnikami pierwszymi danej liczby są $2,\,3$ i $5$. Zatem
			$$\varphi(180)=180\cdot\frac12\cdot\frac23\cdot\frac45=48.$$
			\item $12\,936=2^3\cdot3\cdot7^2\cdot11$, więc dzielnikami pierwszymi danej liczby są $2$, $3$, $7$ i~$11$. Zatem $$\varphi(12\,936)=12\,936\cdot\frac12\cdot\frac23\cdot\frac67\cdot\frac{10}{11}=3360.$$
		\end{itemize}		
	\end{exampleblock}
	
\end{frame}



\begin{frame}{Współczynniki rozkładu silni}

\begin{block}{Twierdzenie}
Niech $n$ będzie liczbą całkowitą dodatnią i niech $\alpha_p(N)$ oznacza największą potęgę liczby $p$ dzielącą liczbę $N$. Wtedy
$$\alpha_p(n!)=\left\lfloor\frac{n}p\right\rfloor+\left\lfloor\frac{n}{p^2}\right\rfloor+\left\lfloor\frac{n}{p^3}\right\rfloor+\ldots=\sum\limits_{k=1}^\infty\left\lfloor\frac{n}{p^k}\right\rfloor.$$
\end{block}

\begin{exampleblock}{Przykład}
Wskaż największą potęgę liczby $3$ dzieląca liczbę $100!$

\bigskip

Korzystając z powyższego wzoru mamy
\begin{align*}
\alpha_3(100!)=&\,\left\lfloor\frac{100}{3}\right\rfloor+\left\lfloor\frac{100}{9}\right\rfloor+\left\lfloor\frac{100}{27}\right\rfloor+\left\lfloor\frac{100}{81}\right\rfloor+\left\lfloor\frac{100}{243}\right\rfloor+\ldots=\\
=&\,33+11+3+1+0+\ldots=48.
\end{align*}
Zatem szukana potęga to $3^{48}$.
\end{exampleblock}


\end{frame}

\begin{frame}

\begin{block}{Twierdzenie}
Dla każdej liczby pierwszej $p$ i dla każdej liczby całkowitej dodatniej $n$ zachodzi $$\alpha_p(n!)<\frac{n}{p-1}$$
\end{block}

\begin{proof}
$$\alpha_p(n!)<\frac{n}p+\frac{n}{p^2}+\frac{n}{p^3}+\ldots=\frac{n}{p}\left(1+\frac1p+\frac1{p^2}+\ldots\right)=\frac{n}p\cdot\frac{p}{p-1}=\frac{n}{p-1}$$
\end{proof}

\end{frame}








\end{document}