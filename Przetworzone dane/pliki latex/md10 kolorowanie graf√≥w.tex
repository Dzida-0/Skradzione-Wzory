
\documentclass[a4paper,10pt]{beamer}
\usepackage[T1,plmath]{polski}
\usepackage[cp1250]{inputenc}
\usepackage{amssymb}
\usepackage{indentfirst}
\usepackage{graphicx}

\usefonttheme[onlymath]{serif}


\usepackage{ulem} % kolorowe podkreślenia
\usepackage{xcolor} % kolorowe podkreślenia

\usepackage{diagbox}
\usepackage{tasks}




\newcommand{\outdeg}{{\,\rm{outdeg}\,}}
\newcommand{\indeg}{{\,\rm{indeg}\,}}

%\definecolor{green1}{html}{22B14C}

\newcommand{\ouline}[1]{{\color{orange}\uline{{\color{black}#1}}}} % pomarańczowe podkreślenie
\newcommand{\yuline}[1]{{\color{yellow}\uline{{\color{black}#1}}}} % żółte podkreślenie
\newcommand{\buline}[1]{{\color{blue}\uline{{\color{black}#1}}}} % niebieskie podkreślenie
\newcommand{\guline}[1]{{\color[RGB]{34,177,76}\uline{{\color{black}#1}}}} % zielone podkreślenie


\usetheme{Boadilla}
\usecolortheme{crane}
%\usecolortheme[rgb={1,0.5,0}]{structure}

\title{\bf Kolorowanie grafów}
%\subtitle{Matematyka, Kierunek: Architektura}
\author[B. Pawlik]{\bf dr inż. Bartłomiej Pawlik}
%\institute{}



%\setbeamercovered{transparent} % przezroczyste warstwy





\begin{document}

\begin{frame}
\titlepage
\end{frame}

\begin{frame}

\begin{block}{Definicja}
{\bf Grafem planarnym} nazywamy graf, który można narysować na płaszczyźnie bez przecięć.
\end{block}


\begin{exampleblock}{Przykład}
Czy graf $K_4$ jest grafem planarnym?

\medskip

Tak (reprezentacja po prawej).

\begin{center}
	
\end{center}
\end{exampleblock}


\begin{exampleblock}{Przykład}
Czy graf $K_{2,3}$ jest grafem planarnym?
\end{exampleblock}



\end{frame}


\begin{frame}


\begin{block}{Twierdzenie}
Grafy $K_{3,3}$ i $K_5$ nie są grafami planarnymi.
\end{block}

\begin{block}{Szkic dowodu.}
Każda reprezentacja graficzna $K_{3,3}$ musi zawierać cykl długości $6$, a każda reprezentacja graficzna $K_{5}$ musi zawierać cykl długości $5$. Wystarczy narysować te cykle i rozpatrzyć wszystkie przypadki dorysowania pozostałych krawędzi.\hfill\qed
\end{block}

\medskip

W teorii grafów planarnych centralną rolę ogrywa {\it twierdzenie Kuratowskiego}. Nie wchodząc w szczegóły, orzeka ono że dany graf nie jest planarny, jeżeli w pewnym sensie {\it zawiera} podgraf podobny do $K_{3,3}$ lub do $K_5$.

\end{frame}


\begin{frame}{Kolorowanie grafów prostych}

	\begin{block}{Definicja}
		Niech $G$ będzie grafem i niech dla pewnej liczby całkowitej $k$, zbiór $C$ będzie zbiorem $k$-elementowym. Funkcję $c:\,V(G)\to C$ nazywamy {\bf $k$-kolorowaniem} grafu $G$, zbiór $C$ nazywamy {\bf zbiorem kolorów}, a elementy zbioru $C$ --- {\bf kolorami}.
	\end{block}

\medskip

Często przyjmuje się, że $C=\{1,2,\ldots,k\}$.

\end{frame}


\begin{frame}

\begin{exampleblock}{Przykład 1}
Rozważmy graf $G$ (na rysunku po lewej) i kolorowanie $c$ ze zbiorem kolorów $C=\{1,\,2,\,3\}$ takie, że
$$c(v_1)=c(v_2)=c(v_4)=1,\ c(v_3)=c(v_5)=2,\ c(v_6)=3.$$
Graf $G$ z zadanym kolorowaniem $c$ możemy przedstawić graficznie na kilka sposobów.
\begin{itemize}
\item Jeżeli w rozważanym kontekście opis wierzchołków jest nieistotny, to wierzchołki możemy indeksować kolorami (rysunek środkowy).
\item Przymując, że kolor $1$ to czerwony, kolor $2$ to zielony, a kolor $3$ to niebieski, wierzchołki możemy (nomen omen) pokolorować (rysunek po prawej). W~tej konwencji można zachować opis wierzchołków.
\end{itemize}

\begin{center}

\end{center}

\end{exampleblock}

\end{frame}


\begin{frame}	
\begin{block}{Definicja}
		Niech $G$ będzie grafem i $c$ będzie $k$-kolorowaniem grafu $G$. Kolorowanie $c$ nazywamy {\bf właściwym $k$-kolorowaniem} grafu $G$, jeżeli dla każdej pary sąsiednich wierzchołków przyjmuje ono róźne wartości:
$$\forall_{u,v\in V(G)}:\ \{u,v\}\in E(G)\ \Rightarrow\ c(u)\neq c(v).$$
\end{block}

\bigskip

\begin{block}{Definicja}
Graf jest {\bf $k$-kolorowalny}, gdy istnieje właściwe $k$-kolorowanie tego grafu.
\end{block}

\end{frame}


\begin{frame}
\begin{exampleblock}{Przykład 2}
Zauważmy, że kolorowanie $c$ w przykładzie 1 nie jest kolorowaniem właściwym, ponieważ występuje w nim dwie para sąsiednich wierzchołków (np. $\{v_1,v_2\}$) mających przypisany ten sam kolor.

\medskip

Przykładowymi kolorowaniami właściwymi grafu $G$ z przykładu 1 są:

\begin{center}

\end{center}

Na rysunkach mamy przedstawione właściwe 6-kolorowanie (po lewej), właściwie 3-kolorowanie (w środku) i właściwe 2-kolorowanie (po prawej) grafu $G$.

\end{exampleblock}

\end{frame}


\begin{frame}
	\begin{alertblock}{Uwaga!}
		W dalszej części wykładu pisząc o {\bf kolorowaniu} ({\bf $k$-kolorowaniu}) będziemy mieli na myśli wyłącznie {\bf kolorowanie właściwe} ({\bf $k$-kolorowanie właściwe}).
	\end{alertblock}

\bigskip

\begin{block}{Definicja}
Kolorowanie, które każdemu wierzchołkowi przyporządkowuje unikalny kolor nazywamy {\bf kolorowaniem naiwnym}.
\end{block}

\bigskip

\begin{exampleblock}{Przykład}
Rysunek po lewej stronie w przykładzie 2 przedstawia przykładowe kolorowanie naiwne rozpatrywanego grafu. 
\end{exampleblock}
	
\end{frame}





\begin{frame}
	
	\begin{block}{Definicja}
		{\bf Liczbą chromatyczną} $\chi(G)$ grafu prostego $G$ nazywamy najmniejszą liczbę $k$ taką, że istnieje kolorowanie $c:V(G)\to\{1,2,\ldots,k\}$.
	\end{block}

\medskip

	Nietrudno zauważyć, że dla dowolnego grafu prostego $G$ zachodzą nierówności
	$$1\leq \chi(G)\leq |V(n)|.$$

\medskip

	\begin{exampleblock}{Przykład}
		Wyznaczyć liczby chromatyczne grafów $P_n$ i $C_n$ dla każdego $n$.
	\end{exampleblock}

\end{frame}



\begin{frame}

	\begin{block}{Stwierdzenie}
		\begin{itemize}
			\item $\chi(G)=1$ wtedy i tylko wtedy, gdy $G$ jest grafem pustym.
			\item $\chi(G)=|V(G)|$ wtedy i tylko wtedy, gdy $G$ jest grafem pełnym.
		\end{itemize}
	\end{block}

\medskip

	\begin{block}{Stwierdzenie}
		$\chi(G)=2$ wtedy i tylko wtedy, gdy $G$ jest niepustym grafem dwudzielnym.
	\end{block}

\medskip

	\begin{block}{Wniosek}
		$\chi(G)\geqslant3$ wtedy i tylko wtedy, gdy $G$ zawiera cykl długości nieparzystej.
	\end{block}

\end{frame}


\begin{frame}

\begin{block}{Twierdzenie}%Dowoód ind. Wilson s.111
 	Jeżeli $G$ jest grafem prostym, to $$\chi(G)\leqslant\Delta(G)+1.$$
\end{block}


\begin{proof}
Przeprowadźmy dowód indukcyjny względem liczby wierzchołków. Niech $G$ będzie grafem prostym mającym $n$ wierzchołków i niech $\Delta=\Delta(G)$.

Z grafu $G$ usuwamy wierzchołek $v$ wraz z przylegającymi do niego krawędziami. Graf $G\backslash\{v\}$ ma $(n-1)$ wierzchołków i $\Delta(G\backslash\{v\})\leqslant\Delta$.
Z założenia indukcyjnego wynika, że $\chi(G\backslash\{v\})\leqslant\Delta+1$.

Wykonujemy $(\Delta+1)$-kolorowanie grafu $G\backslash\{v\}$. Dodajemy do grafu wierzchołek $v$ (z przyległymi krawędziami) i nadajemy mu inny kolor niż mają jego sąsiedzi --- możemy to zrobić, bo liczba sąsiadów wierzchołka $v$ nie przekracza $\Delta$. W ten sposób uzyskaliśmy $(\Delta+1)$ kolorowanie grafu $G$.

\end{proof}


\end{frame}


\begin{frame}
\begin{exampleblock}{Przykład}
Określić dla których cykli i dla których grafów pełnych zachodzi $\chi(G)=\Delta(G)+1$.
\end{exampleblock}
\end{frame}


\begin{frame}

\begin{block}{Twierdzenie (Brooks, 1941)}
	Jeżeli $G$ jest spójnym grafem prostym, nie będącym cyklem nieparzystej długości ani grafem pełnym, to $$\chi(G)\leqslant\Delta(G).$$
\end{block}

\end{frame}



\begin{frame}

\begin{block}{Twierdzenie o czterech barwach (Appel, Haken, 1976)}
Jeżeli $G$ jest planarnym grafem prostym, to $$\chi(G)\leqslant4.$$
\end{block}

\only<1>{\begin{center}
	
\end{center}}
\only<2>{\begin{center}
	
\end{center}}
\only<3>{\begin{center}
	
\end{center}}
\only<4>{\begin{center}
	
\end{center}}
\only<5>{\begin{center}
	
\end{center}}

\end{frame}



\begin{frame}

\begin{block}{Definicja}
 Niech $G$ będzie grafem i niech dla pewnej liczby całkowitej $k$, zbiór $C$ będzie zbiorem $k$-elementowym. Funkcję $c':\,E(G)\to C$ nazywamy {\bf $k$-kolorowaniem krawędziowym} grafu $G$, zbiór $C$ nazywamy {\bf zbiorem kolorów}, a elementy zbioru $C$ --- {\bf kolorami}.
\end{block}

\begin{block}{Definicja}
Niech $G$ będzie grafem prostym i $c'$ będzie $k$-kolorowaniem krawędziowym grafu~$G$. Kolorowanie $c$ nazywamy {\bf właściwym $k$-kolorowaniem krawędziowym} grafu $G$, jeżeli dla każdej pary sąsiednich krawędzi przyjmuje ono róźne wartości:
$$\forall_{\{u,v_1\},\{u,v_2\}\in E(G)}:\ v_1\neq v_2\ \Rightarrow\ c'(\{u,v_1\})\neq c'(\{u,v_2\}).$$
\end{block}

	\begin{alertblock}{Uwaga!}
		W dalszej części wykładu pisząc o {\bf kolorowaniu krawędziowym} ({\bf $k$-kolorowaniu krawędziowym}) będziemy mieli na myśli wyłącznie {\bf właściwe kolorowanie krawędziowe} ({\bf właściwe $k$-kolorowanie krawędziowe}).
	\end{alertblock}

\end{frame}



\begin{frame}
	\begin{block}{Definicja}
		{\bf Indeksem chromatycznym} $\chi'(G)$ grafu prostego $G$ nazywamy najmniejszą liczbę $k$ taką, że istnieje kolorowanie krawędziowe $c':E(G)\to\{1,2,\ldots,k\}$.
	\end{block}

	\begin{exampleblock}{Przykład}
		Wyznaczyć indeks chromatyczny grafu $K_{1,n}$ dla każdej liczby całkowitej dodatniej~$n$.
	\end{exampleblock}

	Graf $K_{1,n}$ nazywany bywa gwiazdą $S_{n-1}$.

	\begin{block}{Stwierdzenie}
		Dla każdego grafu prostego $G$ zachodzi $$\chi'(G)\geqslant\Delta(G).$$
	\end{block}
\end{frame}



\begin{frame}

\begin{block}{Twierdzenie Vizinga (1964)}
 Jeżeli $G$ jest grafem prostym, to
$$\Delta(G)\leqslant\chi'(G)\leqslant\Delta(G)+1.$$
\end{block}


\end{frame}


\begin{frame}


\begin{block}{Definicja}
\begin{itemize}
\item Jeżeli $\chi'(G)=\Delta(G)$, to $G$ nazywamy grafem klasy I.
\item Jeżeli $\chi'(G)=\Delta(G)+1$, to $G$ nazywamy grafem klasy II.
\end{itemize}
\end{block}

\medskip

Poniższy wynik również jest autorstwa Vadima G. Vizinga.

\begin{block}{Twierdzenie}
Jeżeli graf $G$ jest grafem klasy II, to co najmniej trzy wierzchołki tego grafu mają maksymalny stopień.
\end{block}


\end{frame}


\begin{frame}
\begin{block}{Twierdzenie Erd\H{o}sa-Wilsona (1975)}
Niech $Gr(n)$ oznacza zbiór wszystkich grafów prostych mających $n$ wierzchołków i~niech $Cl_I(n)$ oznacza zbiór wszystkich grafów prostych klasy I mających $n$~wierzchołków. Wtedy
$$\lim_{n\to\infty}\frac{|Cl_I(n)|}{|Gr(n)|}=1.$$ 
\end{block}

\end{frame}






\end{document}