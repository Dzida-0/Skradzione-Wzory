
\documentclass[a4paper,10pt]{beamer}
\usepackage[T1,plmath]{polski}
\usepackage[cp1250]{inputenc}
\usepackage{amssymb}
\usepackage{indentfirst}
\usepackage{graphicx}

\usepackage{multicol}

\usefonttheme[onlymath]{serif}


\usepackage{ulem} % kolorowe podkreślenia
\usepackage{xcolor} % kolorowe podkreślenia

\newcommand{\arcctg}{{\rm{arcctg}\,}}
\newcommand{\arctg}{{\rm{arctg}\,}}
\newcommand{\ddd}{{\,\rm{d}}}
\newcommand{\Int}{{\,\rm{Int}}}

%\definecolor{green1}{html}{22B14C}

\newcommand{\ouline}[1]{{\color{orange}\uline{{\color{black}#1}}}} % pomarańczowe podkreślenie
\newcommand{\yuline}[1]{{\color{yellow}\uline{{\color{black}#1}}}} % żółte podkreślenie
\newcommand{\buline}[1]{{\color{blue}\uline{{\color{black}#1}}}} % niebieskie podkreślenie
\newcommand{\guline}[1]{{\color[RGB]{34,177,76}\uline{{\color{black}#1}}}} % zielone podkreślenie



\usetheme{Boadilla}
\usecolortheme{crane}
%\usecolortheme[rgb={1,0.5,0}]{structure}

\title{\bf Notacja $\mathcal{O}$}
%\subtitle{Matematyka, Kierunek: Architektura}
\author[B. Pawlik]{\bf dr inż. Bartłomiej Pawlik}
%\institute{}



%\setbeamercovered{transparent} % przezroczyste warstwy





\begin{document}




\begin{frame}
\titlepage
\end{frame}


\section{Wstęp}

\subsection{Matematyka dyskretna}

% Czym jest matematyka dyskretna.
% Zeilberger stwierdził, że pojęcie liczby rzeczywistej to oksymoron.
% Promień obserwowalnego Wszechświata to 10^27 metrów (największa możliwa do zaobserwowania odległość). Jeżeli chcemy wyliczyć obwód obserwowalnego Wszechświata z dokładnością do odległości Plancka (10^-35 metra, najmniejsza obserwowalna odległość), to wystarczy znać PI z dokładnością do 62 miejsc po przecinku

\begin{frame}
W całym wykładzie przyjmujemy, że zbiór liczb naturalnych to zbiór liczb całkowitych dodatnich.

\bigskip

Jeżeli $n$ jest liczbą naturalną, to
$$\ldots\leqslant \sqrt[4]n\leqslant \sqrt[3]n\leqslant \sqrt{n}\leqslant n\leqslant n^2\leqslant n^3\leqslant n^4\leqslant \ldots$$
Dużo ogólniej:
\begin{block}{}
Jeżeli $n$ jest liczbą naturalną i $\alpha,\beta$ są liczbami rzeczywistymi takimi, że $0\leqslant\alpha\leqslant\beta$, to
$$n^\alpha\leqslant n^\beta.$$
\end{block}
\medskip
Zauważmy, że jeżeli założymy dodatkowo, że $n>1$, to powyższe nierówności będą ostre.

\end{frame}




\begin{frame}

\begin{block}{}
Dla dowolnej liczby naturalnej $n$ mamy $$n<2^n.$$
\end{block}{}

\begin{proof}
Dla $n=1$ nierówność jest oczywista.

Dla dowolnej liczby naturalnej $n>1$ mamy
$$n=\underbrace{2\cdot\frac32\cdot\frac43\cdot\ldots\cdot\frac{n-1}{n-2}\cdot\frac{n}{n-1}}_{n-1}\leqslant\underbrace{2\cdot2\cdot2\cdot\ldots\cdot2\cdot2}_{n-1}=2^{n-1}<2^n.$$
\end{proof}
Zauważmy, że powyżej uzasadniliśmy mocniejszą nierówność
$$n\leqslant2^{n-1}$$
dla $n>1$.
\end{frame}


\begin{frame}

\begin{block}{}
Dla dowolnej liczby naturalnej $n>4$ mamy $$n^2<2^n.$$
\end{block}{}

\begin{proof}
Mamy
$$n^2=4^2\cdot\underbrace{\left(\frac54\right)^2\cdot\left(\frac65\right)^2\cdot\ldots\cdot\left(\frac{n-1}{n-2}\right)^2\cdot\left(\frac{n}{n-1}\right)^2}_{n-4}.$$
Zauważmy, że $\displaystyle\left(\frac54\right)^2<2$ i że jest to największy spośród powyższych ułamków. Zatem
$$n^2<4^2\cdot\underbrace{2\cdot2\cdot\ldots\cdot2\cdot2}_{n-4}=4^2\cdot2^{n-4}=2^n.$$
\end{proof}
\end{frame}



\begin{frame}
\begin{block}{}
Dla dowolnej liczby naturalnej $n$ mamy $$\log_2n<n.$$
\end{block}{}

\begin{proof}
Nierówność otrzymujemy po zlogarytmowaniu stronami wyrażenia $n<2^n$, które wcześniej udowodniliśmy.
\end{proof}

\medskip

Z powyższej nierówności wynika, że dla dowolnej liczby dodatniej $\alpha$ mamy $$\log_2n^\alpha<n^\alpha.$$ W szególności $$\log_2\sqrt{n}<\sqrt{n}.$$
\end{frame}

\begin{frame}
\begin{block}{}
Dla dowolnej liczby dodatniej $\alpha$ i dla dostatecznie dużych wartości $n$ zachodzi $$\log_2n^\alpha<n.$$
\end{block}{}

\begin{proof}
Mamy $$\log_2n^\alpha=\alpha\cdot\log_2n=2\alpha\cdot\frac12\log n=2\alpha\cdot\log\sqrt{n}<2\alpha\cdot\sqrt{n}=\frac{2\alpha}{\sqrt{n}}\cdot n.$$
Zauważmy że dla $n>4\alpha^2$ zachodzi
$$\frac{2\alpha}{\sqrt{n}}\cdot n<\frac{2\alpha}{\sqrt{4\alpha^2}}\cdot n=n,$$
więc ostatecznie dla $n>4\alpha^2$ mamy $\log_2n^\alpha<n.$
\end{proof}
\end{frame}


\begin{frame}

Z nierówności $\log_2n^\alpha<n$ dla $n>4\alpha^2$ wynikają następujące fakty:

\bigskip

\begin{block}{}
Dla dowolnej liczby dodatniej $\alpha$ zachodzą nierówności
$$n^\alpha<2^n\ \ \ \ \mbox{ oraz }\ \ \ \ \log_2n<\sqrt[\alpha]{n}$$
dla dostatecznie dużych wartości $n$.
\end{block}

\bigskip

Reasumując:

\begin{itemize}
\item $2^n$ rośnie szybciej niż jakakolwiek potęga z $n$
\item $\log_2n$ rośnie wolniej niż jakikolwiek pierwiastek z $n$
\end{itemize}

\bigskip

Zatem dla dowolnego $\alpha>1$ mamy
$$\log_2n<\sqrt[\alpha]n<n<n^\alpha<2^n$$
dla dostatecznie dużych $n$.
\end{frame}




\begin{frame}
\begin{block}{}
Nierówność $$2^n<n!$$ zachodzi dla każdej liczby naturalnej $n>3$.
\end{block}{}

\begin{proof}
Mamy $2^4<4!$ oraz
$$2^n=2^4\cdot\underbrace{2\cdot2\cdot\ldots\cdot2\cdot2}_{n-4}<4!\cdot\underbrace{2\cdot2\cdot\ldots\cdot2\cdot2}_{n-4}<4!\cdot\underbrace{5\cdot6\cdot\ldots\cdot(n-1)\cdot n}_{n-4}=n!$$ 
\end{proof}
\end{frame}


\begin{frame}
\begin{block}{}
Nierówność $$100^n<n!$$ zachodzi dla każdej dostatecznie dużej liczby naturalnej $n$.
\end{block}{}

Powyższą nierówność można udowodnić podobnie jak poprzednią $(2^n<n!)$, znajdując najmniejszą liczbę $k$ taką, że $100^k<k!$ i przeprowadzić szacowanie lub indukcję. Poniżej pokażemy dowód nie odwołujący się do poszukiwania tej najmniejszej liczby.

\begin{proof}
Zauważmy, że dla $n>200$ mamy
\begin{align*}
n!>&\,\underbrace{201\cdot202\cdot\ldots\cdot(n-1)\cdot n}_{n-200}>\underbrace{200\cdot200\cdot\ldots\cdot200\cdot200}_{n-200}=200^{n-200}=\\
&\,100^n\cdot2^n\cdot\frac1{200^{200}}=100^n\cdot\frac{2^n}{200^{200}}.
\end{align*}
Oczywiście $2^n$ jest funkcją rosną i nieograniczoną z góry, więc począwszy od pewnego $n$ mamy $2^n>200^{200}$, więc $\displaystyle100^n\cdot\frac{2^n}{200^{200}}>100^n,$ co kończy dowód.
\end{proof}
\end{frame}



\begin{frame}
Analogicznie możemy pokazać, że dla każdej liczby dodatniej $C$ mamy
$$C^n<n!$$
dla dostatecznie dużych $n$, co oznacza że 

\begin{itemize}
\item $n!$ rośnie szybciej niż jakikolwiek ciąg geometryczny
\end{itemize}

\begin{block}{}
Dla dowolnej liczby naturalnej $n>1$ mamy $$n!<n^n$$
\end{block}{}

\begin{proof}
$$n!=1\cdot2\cdot\ldots\cdot(n-1)\cdot n<n\cdot n\cdot\ldots\cdot n\cdot n=n^n$$ 
\end{proof}
\end{frame}

\begin{frame}

\begin{block}{}
$$ \log_2n<n<2^n<n!<n^n$$
\end{block}

\bigskip

Precyzyjniej:
\begin{block}{}
$$1<\ldots<\log_3n<\log_2n<\ldots<\sqrt[3]n<\sqrt{n}<n<n^2<n^3<\ldots$$
oraz
$$\ldots<\sqrt[3]n<\sqrt{n}<n<n^2<n^3<\ldots<2^n<3^n<\ldots<n!<n^n$$
\end{block}

\bigskip

Oczywiście prawdziwe są również nierówności typu
$$n<n\sqrt n<n^2,\ \ \ n<n\cdot\log_2n<n^2,$$
itp.
\end{frame}



\begin{frame}

\begin{block}{Definicja}
Niech $f$ i $g$ będą ciągami liczb rzeczywistych. Przyjmujemy, że $$f_n=\mathcal{O}(g_n)$$ gdy istnieje dodatnia stała $C$ taka, że
$$|f_n|<C\cdot|g_n|$$ dla dostatecznie dużych $n$.
\end{block}

\bigskip

Wyrażenie ,,$f_n=\mathcal{O}(g_n)$'' czytamy ,,{\it $f_n$ jest $O$ od $g_n$}''.

\end{frame}


\begin{frame}
\begin{exampleblock}{Przykład 1}
Z prezentowanych wcześniej nierówności wynika, że
\begin{multicols}{3}
\begin{itemize}
\item $\sqrt{n}=\mathcal{O}(n),$
\item $n=\mathcal{O}(n^2),$
\item $n=\mathcal{O}(2^{n-1}),$
\item $n=\mathcal{O}(2^n),$
\item $n^2=\mathcal{O}(2^n),$
\item $2^n=\mathcal{O}(n!),$
\item $200^n=\mathcal{O}(n!),$
\item $n!=\mathcal{O}(n^n),$
\item $n\log_2{n}=\mathcal{O}(n^2),$
\end{itemize}
\end{multicols}
itp.
\end{exampleblock}

\bigskip

Notacja $O$ służy do szacowania szybkości wzrostu rozpatrywanego ciągu poprzez porównanie ją z szybkością wzrostu prostszego (dobrze znanego) ciągu.
\end{frame}


\begin{frame}
\begin{exampleblock}{Przykład 2}%najmniejsze n to 6
Rozpatrzmy ciąg $$2n^5+9n^3+2024.$$ Dla dużych $n$ wartość wyrażenia $n^5$ jest dużo większa niż wartość wyrażenia $9n^3+2024$, zatem dla dostatecznie dużych $n$ mamy
$$2n^5+9n^3+2024<2n^5+n^5=3n^5.$$
Zatem $$2n^5+9n^3+2024=\mathcal{O}(n^5).$$
\end{exampleblock}

{\bf Uwaga!} Zauważmy, że możemy również szacować:
\begin{itemize}
\item $2n^5+9n^3+2024=\mathcal{O}(n^6),$
\item $2n^5+9n^3+2024=\mathcal{O}(n^5\log_2n),$
\item $2n^5+9n^3+2024=\mathcal{O}(n!),$
\end{itemize}
itp., ale zaproponowane w powyższym przykładzie $\mathcal{O}(n^5)$ jest dużo precyzyjniejszą informacją.
\end{frame}


\begin{frame}
\begin{exampleblock}{Przykład 3 (1/2)}
Rozpatrzmy ciąg $h_n=1+\frac12+\ldots+\frac1n$ dla $n\geqslant1$. Pokażemy, że $h_n=\mathcal{O}(\log_2n)$.

\medskip

Zauważmy, że
\begin{align*}
h_2=&\,1+\frac12<2,\\
h_4=&\,h_2+\left(\frac13+\frac14\right)<2+\left(\frac12+\frac12\right)=2+1=3,\\
h_8=&\,h_4+\left(\frac15+\frac16+\frac17+\frac18\right)<3+\left(\frac14+\frac14+\frac14+\frac14\right)=3+1=4,\\
\intertext{itd. Ogólnie mamy}
h_{2^k}<&\,k+1,
\end{align*}
co można łatwo uzasadnić indukcyjnie.
\end{exampleblock}
\end{frame}


\begin{frame}
\begin{exampleblock}{Przykład 3 (2/2)}
 Niech $n$ będzie liczbą ograniczoną kolejnymi potęgami dwójki: $2^k<n\leqslant2^{k+1}$. Zauważmy, że pierwsza z tych nierówność daje nam $k<\log_2n$. Mamy zatem
 $$h_n\leqslant h_{2^{k+1}}<(k+1)+1=k+2<\log_2n+2.$$
 Dla dostatecznie dużych $n$ mamy $$\log_2n+2<\log_2n+\log_2n=2\log_2n,$$ więc ostatecznie $$h_n=\mathcal{O}(\log_2n).$$
\end{exampleblock}
\end{frame}


\begin{frame}
\begin{block}{Notacja $\mathcal{O}$ --- własności}
\begin{enumerate}
\item Jeżeli $f_n=\mathcal{O}(a_n)$ i $c$ jest stałą, to $$c\cdot f_n=\mathcal{O}(a_n).$$
\item Jeżeli $f_n=\mathcal{O}(a_n)$ i $g_n=\mathcal{O}(a_n)$, to $$f_n+g_n=\mathcal{O}(a_n).$$
\item Jeżeli $f_n=\mathcal{O}(a_n)$ i $g_n=\mathcal{O}(b_n)$, to
$$f_n+g_n=\mathcal{O}(\max\{|a_n|,|b_n|\})\ \mbox{ oraz }\ f_n\cdot g_n=\mathcal{O}(a_n\cdot b_n).$$
\item Jeżeli $a_n=\mathcal{O}(b_n)$ i $b_n=\mathcal{O}(c_n)$, to $$a_n=\mathcal{O}(c_n).$$
\end{enumerate}
\end{block}

\bigskip

(Zauważmy, że powyższe własności pozwalają nam szybko ustalić szacowanie w~przykładzie 2: mamy $2n^5+9n^3+2024=\mathcal{O}(n^5).$)
\end{frame}



\begin{frame}
\begin{block}{Dowód {\it(1/3)}.}
\begin{enumerate}
\item Jeżeli $f_n=\mathcal{O}(a_n)$, to istnieje stała $C>0$ taka, że $|f_n|\leqslant C\cdot|a_n|$ dla dostatecznie dużych $n$. Mamy
$$|c\cdot f_n|=|c|\cdot|f_n|\leqslant|c|\cdot C\cdot |a_n|=(|c|\cdot C)\cdot|a_n|$$ dla dostatecznie dużych $n$, więc $c\cdot f_n=\mathcal{O}(a_n)$.
\item Jeżeli $f_n=\mathcal{O}(a_n)$ oraz $g_n=\mathcal{O}(a_n)$, to istnieją dodatnie stałe $C$ i $D$ takie, że
$$|f_n|\leqslant C\cdot|a_n|\ \ \ \mbox{ oraz }\ \ \ |g_n|\leqslant D\cdot |a_n|$$ dla dostatecznie dużych $n$. W poniższym szacowaniu skorzystamy z nierówności trójkąta $|x+y|\leqslant|x|+|y|$ dla dowonych $x,y\in\mathbb{R}$. Mamy
$$|f_n+g_n|\leqslant|f_n|+|g_n|\leqslant C\cdot|a_n|+D\cdot|a_n|=(C+D)\cdot|a_n|,$$
więc $f_n+g_n=\mathcal{O}(a_n)$.
\end{enumerate}
\end{block}
\end{frame}



\begin{frame}
\begin{block}{Dowód {\it(2/3)}.}
\begin{enumerate}
  \setcounter{enumi}{2}
\item Jeżeli $f_n=\mathcal{O}(a_n)$ oraz $g_n=\mathcal{O}(b_n)$, to istnieją dodatnie stałe $C$ i $D$ takie, że
$$|f_n|\leqslant C\cdot|a_n|\ \ \ \mbox{ oraz }\ \ \ |g_n|\leqslant D\cdot |b_n|$$ dla dostatecznie dużych $n$. Ponownie korzystając z nierówności trójkąta mamy 
\begin{align*}
|f_n+g_n|\leqslant&\,|f_n|+|g_n|\leqslant C\cdot|a_n|+D\cdot|b_n|\leqslant\\
&\,C\cdot\max\{|a_n|,|b_n|\}+D\cdot\max\{|a_n|,|b_n|\}=\\
&\,(C+D)\cdot\max\{|a_n|,|b_n|\},
\end{align*}
więc $f_n+g_n=\mathcal{O}(\max\{|a_n|,|b_n|\})$. Ponadto
$$|f_n\cdot g_n|=|f_n|\cdot|g_n|\leqslant C\cdot|a_n|\cdot D\cdot|b_n|=(C\cdot D)\cdot|a_n\cdot b_n|,$$
więc $f_n\cdot g_n=\mathcal{O}(a_n\cdot b_n)$.
\end{enumerate}
\end{block}
\end{frame}



\begin{frame}
\begin{proof}[Dowód {\it(3/3)}]
\begin{enumerate}
  \setcounter{enumi}{3}
\item Jeżeli $a_n=\mathcal{O}(b_n)$ oraz $b_n=\mathcal{O}(c_n)$, to istnieją dodatnie stałe $B$ i $C$ takie, że
$$|a_n|\leqslant B\cdot|b_n|\ \ \ \mbox{ oraz }\ \ \ |b_n|\leqslant C\cdot |c_n|$$ dla dostatecznie dużych $n$. Zatem
$$|a_n|\leqslant B\cdot |b_n|\leqslant B\cdot |C\cdot c_n|=(B\cdot C)\cdot|c_n|,$$
więc $a_n=\mathcal{O}(c_n)$.
\end{enumerate}
\end{proof}
\end{frame}








\end{document}