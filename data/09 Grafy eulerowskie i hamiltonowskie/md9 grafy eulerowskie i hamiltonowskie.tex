
\documentclass[a4paper,10pt]{beamer}
\usepackage[T1,plmath]{polski}
\usepackage[cp1250]{inputenc}
\usepackage{amssymb}
\usepackage{indentfirst}
\usepackage{graphicx}

\usefonttheme[onlymath]{serif}


\usepackage{ulem} % kolorowe podkreślenia
\usepackage{xcolor} % kolorowe podkreślenia

\usepackage{diagbox}
\usepackage{tasks}




\newcommand{\outdeg}{{\,\rm{outdeg}\,}}
\newcommand{\indeg}{{\,\rm{indeg}\,}}

%\definecolor{green1}{html}{22B14C}

\newcommand{\ouline}[1]{{\color{orange}\uline{{\color{black}#1}}}} % pomarańczowe podkreślenie
\newcommand{\yuline}[1]{{\color{yellow}\uline{{\color{black}#1}}}} % żółte podkreślenie
\newcommand{\buline}[1]{{\color{blue}\uline{{\color{black}#1}}}} % niebieskie podkreślenie
\newcommand{\guline}[1]{{\color[RGB]{34,177,76}\uline{{\color{black}#1}}}} % zielone podkreślenie


\usetheme{Boadilla}
\usecolortheme{crane}
%\usecolortheme[rgb={1,0.5,0}]{structure}

\title{\bf Grafy eulerowskie i hamiltonowskie}
%\subtitle{Matematyka, Kierunek: Architektura}
\author[B. Pawlik]{\bf dr inż. Bartłomiej Pawlik}
%\institute{}



%\setbeamercovered{transparent} % przezroczyste warstwy





\begin{document}


\begin{frame}
\titlepage
\end{frame}

\begin{frame}
\begin{block}{Drzewa przeszukiwań}
\begin{itemize}
\item {\bf Metoda przeszukiwania wszerz} ({\bf breadth-first search}, {\bf BFS})

Odwiedzamy wszystkich sąsiadów aktualnego wierzchołka, zanim przejdziemy do następnego wierzchołka.

\item {\bf Metoda przeszukiwania w głąb} ({\bf deapth-first search}, {\bf DFS})

Po odwiedzeniu sąsiada $v_{k+1}$ wierzchołka $v_k$, przechodzimy do nieodwiedzonego sąsiada $v_{k+2}$ wierzchołka $v_{k+1}$ albo --- w przypadku braku nieodwiedzony sąsiadów ---  cofamy się do wierzchołka $v_k$ i powtarzamy.
\end{itemize}
\end{block}

\medskip

W {\it BFS} zanim zagłębimy się bardziej w drzewo, zagładamy do tak wielu wierzchołków jak to możliwe, a w {\it DFS} zagłębiamy się tak daleko, jak tylko jest to możliwe, zanim zajrzymy do innych wierzchołków sąsiednich.
\end{frame}


\begin{frame}

\begin{exampleblock}{Przykład}
	\begin{center}
		
	\end{center}
Przykładowe wyniki algorytmów przeszukiwania wszerz i w głąb, zaczynając od wierzchołka $a$:
$$BFS:\ a,\,b,\,g,\,c,\,d,\,e,\,h,\,f,\,i,\,j$$
$$DFS:\ a,\,b,\,c,\,d,\,f,\,e,\,g,\,h,\,i,\,j$$
Jakie wyniki może zwrócić algorytm, gdy zaczniemy od wierzchołka $d$?
\end{exampleblock}

\end{frame}


\begin{frame}
	
\begin{block}{Definicja}
	Niech $G=(V(G),E(G))$ będzie grafem i niech $w:\,E(G)\to\mathbb{R}$ będzie funkcją. Parę $(G,w)$ nazywamy {\it grafem ważonym}.
\end{block}

\medskip

Innymi słowy, grafem ważonym nazywamy graf, w którym każej krawędzi przypisana jest liczba rzeczywista (może ona reprezentować odległość między wierzchołkami, przepustowość sieci, ilość interakcji itd.).

\end{frame}




\begin{frame}{Szukanie najkrótszej drogi - algorytm Dijkstry}
	
	Algorytm służy do wyszukiwania najkrótszej drogi od danego wierzchołka do pozostałych w grafie ważonym bez pętli, w którym wagi są liczbami \underline{nieujemnymi}.
	
	\begin{block}{}
		Niech $V(G)=\{0,1,2,\ldots,n-1\}$. Będziemy szukać najkrótszej drogi od wierzchołka $0$ do pozostałych.
		
		{\bf Krok 1:} $Q:=\{0,1,\ldots,n\}$, $S:=\emptyset$,
		
		$d(0):=0$, $p(0):=0$ oraz $d(v):=\infty$, $p(v):=\infty$ dla $1\leq v\leq n$.
		
		{\bf Krok 2:} Dopóki $Q\neq\emptyset$ powtarzaj:
		\begin{enumerate}
			\item Wybierz $v\in Q$ o najmniejszej wartości $p(v)$.
			\item $Q:=Q/\{v\}$, $S:=S\cup\{v\}$.
			\item Dla każdego wierzchołka $u$ sąsiedniego z $v$ dokonaj {\bf relaksacji}, tzn. jeśli $d(u)>d(v)+w(v,u)$, to $d(u):=d(u)+w(v,u)$ oraz $p(u)=v$.
		\end{enumerate} 
		{\bf Krok 3:} Minimalna długość drogi od $0$ do $v$ to $d(v)$. Jeśli $d(v)=\infty$ to nie istnieje droga od $0$ do $d(v)$ (to jest możliwe w digrafie).
	\end{block}
	
\end{frame}


\begin{frame}
	
	\begin{exampleblock}{Przykład}
		W poniższym grafie znaleźć najkrótsze drogi z wierzchołka $B$ do pozostałych wierzchołków korzystając z algorytmu Dijkstry.
		
		\begin{minipage}{.45\textwidth}
			\begin{center}
				
			\end{center}
		\end{minipage}
		\begin{minipage}{.45\textwidth}
				$$\begin{array}{|c|c|c|c|c|c|c|}\cline{2-7}			
					\multicolumn{1}{c|}{}&A&C&D&E&F&G\\\hline
					B_0&6_B&\infty&\textcolor{red}{1_B}&\infty&\infty&7_B\\\hline
					BD_1&6_B&\textcolor{red}{4_D}&-&\infty&5_D&7_B\\\hline
					BDC_4&\textcolor{magenta}{5_C}&-&-&9_C&\textcolor{red}{5_D}&7_B\\\hline
					BDF_5&\textcolor{red}{5_C}&-&-&8_F&-&6_F\\\hline
					BDCA_5&-&-&-&8_F&-&\textcolor{red}{6_F}\\\hline
					BDFG_6&-&-&-&\textcolor{red}{8_F}&-&-\\\hline
					BDFE_8&-&-&-&-&-&-\\\hline
				\end{array}$$
		\end{minipage}\\
	\vspace{0.3cm}
	W pierwszej kolumnie zapisane są najkrótsze drogi (w indeksie każdej z nich jest waga).
	
	Elementy tabeli to wagi dróg z wierzchołka $B$ do danego wierzchołka (w~indeksach są wierzchołki bezpośrednio poprzedzające dany).
	
	Symbol ,,$-$'' oznacza, że do danego wierzchołka już określono najkrótszą drogę.
	
	W sytuacji, gdy w wierszu mamy więcej niż jedną drogę o najmniejszej wadze ($\textcolor{magenta}{5_C}$ do $A$ i $\textcolor{red}{5_D}$ do $F$ w wierszu czwartym), wybieramy dowolną z nich.
		
	\end{exampleblock}
	
\end{frame}

\begin{frame}

Zauważmy, że problem najkrótszej drogi można rozwiązać również algorytmem {\it brute force} --- wystarczy określić wagę wszystkich ścieżek\footnote{Oczywiście najkrótsza droga musi być ścieżką.} między rozważanymi wierzchołkami i wybrać najmniejszą. Jednak już nawet dla grafów o~niewielkiej liczbie wierzchołków ten algorytm jest dużo mniej optymalny.

\medskip

Zauważmy jednak, że znalezienie jakiejkolwiek ścieżki między rozważanymi wierzchołkami daje w oczywisty sposób \underline{ograniczenie górne} na rozwiązanie problemu najkrótszej drogi.

\end{frame}



\begin{frame}{Grafy eulerowskie}

	\begin{block}{Definicja}
\begin{itemize}
	\item	Jeżeli w grafie $G$ istnieje cykl niewłaściwy $d$ przechodzący przez każdą krawędź grafu $G$ dokładnie jeden raz, to $d$ nazywamy {\bf cyklem Eulera}, a graf $G$ --- {\bf grafem eulerowskim}.
	\item Jeżeli graf $G$ nie jest grafem eulerowskim i istnieje ścieżka $d$ przechodząca przez każdą krawędź grafu $G$ dokładnie jeden raz, to $G$ nazywamy {\bf grafem jednobieżnym} ({\bf półeulerowskim}).
\end{itemize}
	\end{block}

\medskip

\begin{exampleblock}{Przykład}
		\begin{minipage}{.6\textwidth}
			\begin{center}
		
			\end{center}
		\end{minipage}
\hfill
		\begin{minipage}{.35\textwidth}
$G_1$ --- graf eulerowski

\medskip

$G_2$ --- graf jednobieżny

\medskip

$G_3$ --- graf, który nie jest ani eulerowski, ani jednobieżny
		\end{minipage}
\end{exampleblock}
\end{frame}



\begin{frame}

\begin{block}{Problem mostów królewieckich}
Czy można przejść dokładnie jeden raz przez każdy z siedmiu mostów przedstawionych na poniższym rysunku i powrócić do punktu wyjścia?
\end{block}
\begin{center}
	

\begin{footnotesize}[Źródło: {\it https://pl.wikipedia.org/wiki/Królewiec\#/media/Plik:Image-Koenigsberg,\_Map\_by\_Merian-Erben\_1652.jpg}, 13.05.2024]\end{footnotesize}
\end{center}

\end{frame}




\begin{frame}
	\begin{block}{Lemat 1}
Jeżeli $\delta(G)\geqslant2$, to graf $G$ zawiera cykl.
	\end{block}
\begin{proof}
Jeżeli $G$ zawiera pętle lub krawędzie wielokrotne, to lemat jest trywialny. Załóżmy zatem, że $G$ jest grafem prostym i że $v\in G$ jest dowolnym wierzchołkiem $G$.

Tworzymy trasę $$v\to v_1\to v_2\to\ldots$$
tak aby $v_{k+1}$ był sąsiadem $v_k$ różnym od $v_{k-1}$ (zawsze jest to możliwe ze względu na założenie, że stopień każdego wierzchołka wynosi co najmniej $2$). Graf $G$ ma skończenie wiele wierzchołków, więc po skończonej liczbie kroków na trasie musi pojawić się wierzchołek $v_K$, który pojawił się na niej już wcześniej. Fragment trasy $$v_K\to\ldots\to v_K$$ jest szukanym cyklem.
\end{proof}	
\end{frame}





\begin{frame}
	\begin{block}{Twierdzenie (Euler, 1741)}
Niech $G$ będzie grafem spójnym. Graf $G$ jest eulerowski wtedy i tylko wtedy, gdy każdy wierzchołek ma stopień parzysty.
	\end{block}
\begin{block}{Dowód. {\it (1/2)}}
($\Rightarrow$)

Niech $d$ będzie cyklem Eulera w grafie $G$ i niech $v\in V(G)$. Każde przejście cyklu $d$ przez wierzchołek $v$ zwiększa udział krawędzi należących do tego cyklu w stopniu wierzchołka $v$ o $2$. Każda krawędź występuje w grafie $G$ dokładnie raz, więc stopień każdego wierzchołka musi być liczbą parzystą.
\end{block}	
\end{frame}

\begin{frame}

\begin{block}{Dowód {\it (2/2)}}
($\Leftarrow$)

Dowód indukcyjny względem liczby krawędzi grafu $G$. $G$ jest grafem spójnym, w~którym stopień każdego wierzchołka jest liczbą parzystą, więc $\delta(G)\geqslant2$. Zatem, na mocy Lematu 1, w $G$ istnieje cykl $c$.
\begin{itemize}
\item Jeżeli $c$ zawiera każdą krawędź grafu $G$, to twierdzenie jest udowodnione.
\item Jeżeli $c$ nie zawiera każdej krawędzi grafu $G$, to usuwamy z grafu $G$ wszystkie krawędzie należące do cyklu $c$, otrzymując graf $G\backslash\{c\}$, który ma mniej krawędzi niż $G$. Z założenia indukcyjnego każda składowa spójności grafu $G\backslash\{c\}$ posiada cykl Eulera. Ze spólności $G$ wynika, że każda składowa spójności grafu $G\backslash\{c\}$ ma co najmniej jeden wierzchołek wspólny z cyklem $c$. Zatem w grafie $G$ można stworzyć cykl Eulera poprzez połączenie cyklu $c$ z~cyklami Eulera wszystkich składowych spójności grafu $G\backslash\{c\}$ poprzez wierzchołki wspólne.
\end{itemize}
\end{block}
\end{frame}


\begin{frame}
	
	\begin{exampleblock}{Przykład}
		\begin{minipage}{.4\textwidth}
			\begin{center}
			
			\end{center}
		\end{minipage}
		\begin{minipage}{.55\textwidth}
				Na powstawie powyższego twierdzenia Eulera określić czy dany graf jest eulerowski lub jednobieżny.

\medskip
\begin{itemize}
\item Jeżeli jest eulerowski, to wskazać przykładowy cykl Eulera.
\item Jeżeli jest jednobieżny to wskazać przykładową ścieżkę Eulera.
\end{itemize}
		\end{minipage}\\
	\vspace{0.3cm}
Dany graf jest eulerowski, ponieważ stopień każdego wierzchołka jest liczbą parzystą. Przykładowy cykl Eulera to $$(1\to2\to8\to1)\to(3\to4\to2\to3)\to(5\to6\to4\to5)\to$$ $$\to(7\to8\to6\to7)\to1.$$
	\end{exampleblock}
	
\end{frame}



\begin{frame}

\begin{block}{Wniosek}
Niech $G$ będzie grafem spójnym.
\begin{itemize}
\item Graf $G$ jest grafem eulerowskim wtedy i tylko wtedy, gdy jego zbiór krawędzi można podzielić na rozłączne cykle.
\item Graf $G$ jest grafem jednobieżnym wtedy i tylko wtedy, gdy ma dokładnie dwa wierzchołki nieparzystych stopni.
\end{itemize}
\end{block}

\end{frame}






\begin{frame}{Znajdowanie cyklu Eulera}
	
	\begin{block}{Definicja}
	 {\bf Mostem} nazywamy tę krawędź w grafie skończonym $G$, której usunięcie powoduje zwiększenie liczby spójnych składowych grafu $G$. 
	\end{block}

\begin{exampleblock}{Przykład}
	\begin{center}
	
	\end{center}
Jedynymi mostami w powyższym grafie są krawędzie $\{c,d\}$ oraz $\{f,g\}$.
\end{exampleblock}

\end{frame}

\begin{frame}
	\begin{block}{Znajdowanie cyklu Eulera - algorytm Fleury'ego}
		Zaczynając od dowolnego wierzchołka, tworzymy cykl Eulera dodając do niego kolejne krawędzie w taki sposób, że dodajemy do cyklu most \underline{tylko wtedy}, gdy nie ma innej możliwości.
	\end{block}

	\begin{exampleblock}{Przykład}
		Wyznaczyć cykl Eulera w grafie $K_{2,4}$.
	\end{exampleblock}
	
\end{frame}





\begin{frame}{Problem chińskiego listonosza}
	
	\begin{block}{Sformułowanie}
		Znalezienie cyklu (niewłaściwego) zawierającego każdą krawędź danego grafu co najmniej raz i mającego jak najmniejszy koszt (czyli liczbę krawędzi lub, w~przypadku grafu ważonego, sumę wag krawędzi).
	\end{block}
	
	\begin{block}{Rozwiązanie problemu chińskiego listonosza}
		\begin{itemize}
			\item Jeżeli graf $G$ jest eulerowski to znajdujemy dowolny cykl Eulera.
			\item Jeżeli graf nie jest eulerowski to dublujemy niektóre krawędzie, aby otrzymać graf eulerowski i wtedy szukamy ścieżki Eulera. Aby otrzymać najoptymalniejsze (najkrótsze) ścieżki
			\begin{itemize}
				\item dla grafów nieważonych wystarczy przeszukać graf wszerz,
				\item dla grafów z nieujemnymi wagami można skorzystać np. z algorytmu Dijkstry,
				\item dla grafów z dowolnymi wagami można skorzystać np. z algorytmu Bellmana-Forda.
			\end{itemize}
		\end{itemize}
		
	\end{block}
	
\end{frame}



\begin{frame}
	
	\begin{exampleblock}{Przykład}
Rozwiąż problem chińskiego listonosza dla poniższego grafu. Jaki jest koszt cyklu stanowiącego rozwiązanie?

		\begin{minipage}{.4\textwidth}
			\begin{center}
			\end{center}
		\end{minipage}
		\begin{minipage}{.55\textwidth}
			Wierzchołki $b$ i $f$ są jedynymi wierzchołkami nieparzystego stopnia, więc dany graf jest jednobieżny. Najkrótsza droga z $b$ do $f$ to $$b\to a\to d\to f,$$ więc w grafie dublujemy krawędzie $$\{a,b\},\,\{a,d\}\mbox{ oraz }\{d,f\},$$ dzięki czemu uzyskaliśmy graf eulerowski.
		\end{minipage}\\
	\vspace{0.3cm}
Przykładowym cyklem Eulera w nowym grafie (i zarazem rozwiązaniem problemu chińskiego listonosza) jest
$$a\to e\to h\to b\to a\to d\to f\to b\to a\to d\to$$
$$\to e\to g\to h\to f\to d\to c\to g\to f\to c\to a,$$
którego koszt wynosi $48$.
	\end{exampleblock}
\end{frame}










\begin{frame}{Grafy hamiltonowskie}
	
\begin{block}{Definicja}
\begin{itemize}
	\item	Jeżeli w grafie $G$ istnieje cykl $h$ przechodzący przez każdy wierzchołek grafu $G$ dokładnie jeden raz, to $h$ nazywamy {\bf cyklem Hamiltona}, a graf $G$ --- {\bf grafem hamiltonowskim}.
	\item Jeżeli graf $G$ nie jest grafem hamiltonowskim i istnieje ścieżka $h$ przechodząca przez każdy wierzchołek tego grafu dokładnie jeden raz, to $G$ nazywamy {\bf grafem trasowalnym} ({\bf półhamiltonowskim}).
\end{itemize}
\end{block}

\medskip

\begin{exampleblock}{Przykład}
		\begin{minipage}{.55\textwidth}
			\begin{center}
				
			\end{center}
		\end{minipage}
\hfill
		\begin{minipage}{.4\textwidth}
$G_1$ --- graf hamiltonowski

\medskip

$G_2$ --- graf trasowalny

\medskip

$G_3$ --- graf, który nie jest ani hamiltonowski, ani trasowalny
		\end{minipage}
\end{exampleblock}

\end{frame}

\begin{frame}

\begin{alertblock}{Uwaga!}
Powszechnie przyjęte nazewnictwo: {\bf graf półeulerowski}, {\bf graf półhamiltonowski} jest trochę niefortunne, bo sugeruje, że każdemu z tych grafów dużo brakuje do bycia ,,pełnymi" grafami eulerowskimi lub hamiltonowskimi, podczas gdy w każdym przypadku wystarczy w tym celu do grafu dodać \underline{tylko jedną krawędź}.
\end{alertblock}

\end{frame}

\begin{frame}	
	\begin{block}{Twierdzenie (Ore, 1960)}
		Jeżeli graf prosty $G$ ma $n$ wierzchołków ($n\geqslant3$) oraz $$\deg(u)+\deg(v)\geqslant n$$ dla każdej pary wierzchołków niesąsiednich $u$ i $v$, to graf $G$ jest hamiltonowski.
	\end{block}

\begin{block}{Dowód. ({\it 1/2})}
Załóżmy nie wprost, że dla ustalonego $n\geq3$ istnieją grafy niehamiltonowskie spełniające założenia rozpatrywanego twierdzenia. Niech $G$ będzie takim grafem z~jak największą liczbą krawędzi --- jeżeli do $G$ dołączymy jedną krawędź to otrzymamy graf hamiltonowski.
\end{block}
\end{frame}

\begin{frame}

\begin{block}{Dowód. ({\it 2/2})}
W $G$ istnieje co najmniej jedna para wierzchołków, które nie są połączone krawędzią --- bez straty ogólności przyjmijmy że są to wierzchołki $v_1$ i $v_n$. Z faktu, że po dodaniu krawędzi otrzymalibyśmy cykl Hamiltona wynika, że w grafie $G$ istnieje droga $$v_1\to v_2\to\ldots\to v_{n-1}\to v_n.$$ Z założenia wiemy, że
$$\deg v_1+\deg v_n\geqslant n,$$
co oznacza że istnieje indeks $k$ taki, że $v_{k-1}$ jest sąsiadem $v_n$ i że $v_k$ jest sąsiadem $v_1$. Jednak teraz cykl 
$$v_1\to \ldots\to v_{k-1}\to v_n\to\ldots\to v_k\to v_1$$ jest cyklem Hamiltona, co daje nam szukaną sprzeczność.
\end{block}
	
	
\end{frame}
	
	
\begin{frame}
	\begin{block}{Twierdzenie (Dirac, 1952)}
		Jeżeli minimalny stopień grafu $G$ jest nie mniejszy niż połowa liczby wierzchołków tego grafu:
		$$\delta(G)\geqslant\frac{|V(G)|}2,$$
		 to $G$ jest grafem hamiltonowskim.
	\end{block}
	
	
	\begin{proof}
		Niech $u,v\in V(G)$ i niech $|V(G)|=n$. Każdy wierzchołek grafu $G$ ma stopień nie mniejszy niż $\displaystyle \frac{n}2$, więc $$\deg(u)+\deg(v)\geqslant \frac{n}2+\frac{n}2=n.$$
		Zatem w szczególności sumy stopni wszystkich par \underline{niesąsiednich} wierzchołków są nie mniejsze niż $n$, więc z twierdzenia Orego otrzymujemy, że $G$ jest grafem hamiltonowskim.
	\end{proof}
	
\end{frame}



\begin{frame}
	
	\begin{block}{Problem komiwojażera}
		Podamy dwie wersje problemu:
		\begin{itemize}
		\item Mając daną listę miast i odległości między tymi miastami, znaleźć najkrótszą drogę przechodzącą przez wszystkie miasta (przez każde tylko raz) i~powracającą do punktu wyjścia.	
		\item Znaleźć najoptymalniejszy cykl Hamiltona w ważonym grafie pełnym.
		\end{itemize}
	\end{block}

	\begin{block}{Przykładowe rozwiązania}
		\begin{itemize}
			\item {\it Brute force}: Znajdujemy wszystkie cykle Hamiltona i wybieramy najodpowiedniejszy.
			\item {\it Algorytm najbliższego sąsiada}: Zaczynamy od dowolnego wierzchołka i poruszamy się zawsze wzdłuż krawędzi o najmniejszych wagach. (jest to rozwizanie przybliżone, średnio o $25\%$ gorsze od optymalnego).
		\end{itemize}
	\end{block}

\end{frame}





\end{document}
